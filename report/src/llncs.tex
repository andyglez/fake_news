\documentclass{llncs}

%
\usepackage{geometry} % to change the page dimensions
\usepackage{natbib}


\geometry{a4paper}

\begin{document}


\title{Detecci\'on autom\'atica de noticias falsas}

\author{Andy Gonz\'alez Pe\~na \\ a.gonzalez@estudiantes.matcom.uh.cu}
\institute{Matem\'atica y Computaci\'on, Universidad de la Habana, 2018}

\maketitle


\begin{multicols}{2}[]

\begin{abstract}

Con el auge de las redes sociales el consumo de informaci\'on negativa ha aumentado exponencialmente al punto que se requieren de nuevos mecanismos
que, al menos, regulen su circulaci\'on. Encabezando la lista se encuentran las noticias falsas que, a pesar de siempre haber existido ya sea por fines pol\'iticos
u otros, toman nueva relevancia al aparentar pertenecer a canales oficiales o aparecer en la red de confianza de un usuario de una determinada red social.
En este trabajo se presentan las caracter\'isticas principales del problema antes de la aplicaci\'on en la web que se conoce en la actualidad, es decir, se propone
como reconocer patrones de una noticia desde el punto de vista est\'atico en el que no se puede hacer nada por no recibir tal noticia, todo lo contrario, se recibe
y se decidir\'a la validez a partir de un an\'alisis utilizando t\'ecnicas de \textit{Machine Learning} (ML).

\end{abstract}


\section{Introducci\'on}\label{sec:Introduction}

Muchos esfuerzos son dedicados en la actualidad para enfrentar la sombra que proyectan las divulgaciones falsas y negativas. En este trabajo se resumen las
principales t\'ecnicas existentes para resolver esta problem\'atica as\'i como las diferencias entre los resultados aplicados a modelos que solamente dependen de la
noticia en cuesti\'on como las propuestas enfocadas a las redes sociales, esto se puede encontrar en la secci\'on de \textbf{Estado del Arte}.

Tomando en cuenta el an\'alisis arriba mencionado se proceder\'a a proponer una soluci\'on, en la secci\'on de igual nombre, basada en t\'ecnicas de ML con datos como
titular y texto exclusivamente y utilizando un \textit{corpus} de clasificaci\'on en solamente dos categor\'ias, reales o falsas, utilizando las principales caracter\'isticas del
procesamiento de lenguaje natural (NLP) que no depender\'a de entidades externas como en una plataforma social, es decir, autores o referencias, sino utilizando \'unicamente 
los patrones m\'as abundantes dentro del texto. En consecuencia, se ofrecer\'an los resultados alcanzados y la relevancia de los mismos en comparaci\'on con los sistemas 
ya existentes en explotaci\'on.

Dentro de la rama de miner\'ia de datos, la detecci\'on autom\'atica de noticias falsas se encuentra a\'un en sus inicios y existen muchos trabajos y discusiones abiertas,
por tanto, la propuesta de soluci\'on no deber\'a ser tomada a la ligera ni con el af\'an de implementaci\'on \'optima. De manera concluyente se presentan aspectos te\'oricos
que podr\'ian brindar mejores resultados as\'i como futuras incorporaciones que deber\'an ser aplicadas con el objetivo de proponer un sistema altamente certero.

Con este trabajo se persigue la consolidaci\'on de los conocimientos b\'asicos de miner\'ia de datos, inteligencia artificial y sistemas de informaci\'on en cuesti\'on, la aplicaci\'on 
de esquemas de procesamiento de lenguaje natural (NLP) y abrir nuevos debates constructivos que impacten en un grado positivo sistemas de estas caracter\'isticas para, 
con ello, poder disfrutar en mayor medida de la parte sana de la actual difusi\'on de la informarci\'on.

Sin pretensiones de alta profundizaci\'on en diversas tem\'aticas especialmente aplicadas a las redes sociales, o la propia definici\'on de noticias falsas, como pueden ser factores
 psicol\'ogicos, pol\'iticos, sociales u otros, se procede a continuar con la siguiente secci\'on, ce\~nida a elementos puramente computacionales.
\section{Estado del Arte}
\end{multicols}

\bibliographystyle{apa}
\bibliography{ref}


\end{document}